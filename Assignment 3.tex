\documentclass[12pt]{article}




\usepackage{graphicx}
\graphicspath{{image/}}

\title{my}

\begin{document}


\centering\huge\texttt{NATIONAL INSTITUTE OF TECHNOLOGY, RAIPUR}\\

\begin{figure}[h]
\centering
\includegraphics[scale=1.5]{NITRR.jpg}
\end{figure}

\begin{center}


\huge\texttt{ASSIGNMENT \\ON\\ FUTURE OF HEALTHCARE}

\end{center}

\begin{minipage}[t]{5cm}


\flushleft\Large\textbf{\underline{submitted by:}}\\
Itisha Kaiwart\\
Roll No.-21111024
\end{minipage}
\hfill
\begin{minipage}[t]{5cm}


\Large\textbf{\underline{under the supervision of:}}
Dr. Saurabh Gupta
\end{minipage}
\pagebreak
\tableofcontents
\pagebreak

\section{\centering\Large\texttt{ACKNOWLEDGEMENT}}

\large \flushleft I would like to express my special thanks of gratitude to Dr. Sourabh Gupta who gave me the opportunity to do this assignment on "Disruptive innovation in healthcare".I came to know about so many thingsI am really thankfull to them .\\



\vspace{1cm}
Secondly I would like to thanks my parents and friends who helped me a lot in finalising this assignment within the limited time frame.

\pagebreak


\section{\centering\Large\texttt{INTRODUCTION}}

\vspace{0.5cm}
\large\flushleft The future of healthcare is shaping up in front of our very eyes with advances in digital healthcare technologies, such as artificial intelligence, VR/AR, 3D-printing, robotics or nanotechnology. We have to familiarize with the latest developments in order to be able to control technology and not the other way around. The future of healthcare lies in working hand-in-hand with technology and healthcare workers have to embrace emerging healthcare technologies in order to stay relevant in the coming years. \\

\vspace{0.5cm}

\begin{figure}[h]
\centering
\includegraphics[scale=1.5]{assign.jpg}
\end{figure}

\pagebreak


\section{\Large\centering\texttt{FUTURE OF HEALTH CARE}}

In medicine and healthcare, digital technology could help transform unsustainable healthcare systems into sustainable ones, equalize the relationship between medical professionals and patients, provide cheaper, faster and more effective solutions for diseases – technologies could win the battle for us against cancer, AIDS or Ebola – and could simply lead to healthier individuals living in healthier communities.\\


\vspace{0.5cm}

\textbf{Some technologies which may change the future of healthcare are:}



\begin{itemize}

\item \textbf{ARTIFICIAL INTELLIGENCE}\\
 Artificial intelligence has the potential to redesign healthcare completely. AI algorithms are able to mine medical records, design treatment plans or create drugs way faster than any current actor on the healthcare palette including any medical professional. 

Atomwise uses supercomputers that root out therapies from a database of molecular structures. In 2015, the start-up launched a virtual search for safe, existing medicines that could be redesigned to treat the Ebola virus. They found two drugs predicted by the company’s AI technology which may significantly reduce Ebola infectivity.



\item\textbf{VIRTUAL REALITY}\\
Virtual reality (VR) is changing the lives of patients and physicians alike. In the future, you might watch operations as if you wielded the scalpel or you could travel to Iceland or home while you are lying on a hospital bed. 

VR is being used to train future surgeons and for actual surgeons to practice operations. Such software programmes are developed and provided by companies like Osso VR and ImmersiveTouch and are in active use with promising results

\item\textbf{AUGMENTED REALITY}\\
Augmented reality differs from VR in two respects: users do not lose touch with reality and it puts information into eyesight as fast as possible. These distinctive features enable AR to become a driving force in the future of medicine; both on the healthcare providers’ and the receivers’ side.



\item\textbf{3D PRINTING}\\
3D-printing can bring wonders in all aspects of healthcare. We can now print biotissues, artificial limbs, pills, blood vessels and the list goes on and will likely keep on doing so.\\

In November 2019, researchers at the Rensselaer Polytechnic Institute in Troy, New York, developed a method to 3D-print living skin along with blood vessels. This development proves crucial for skin grafts for burn victims. Also, helping patients in need are NGOs like Refugee Open Ware and Not Impossible which 3D-print prosthetics for refugees from war-torn areas. 





\item\textbf{ROBOTICS}\\
One of the most exciting and fastest growing fields of healthcare is robotics; developments range from robot companions through surgical robots until pharmabotics, disinfectant robots or exoskeletons. 

2019 was a great year for exoskeletons. It saw Europe’s first exoskeleton-aided surgery and a tetraplegic man capable of controlling an exoskeleton with his brain! There are loads of other applications for these sci-fi suits from aiding nurses through lift elderly patients to helping patients with spinal cord injury.



\item\textbf{HEALTHCARE TRACLERS, WEARABLE AND SENSORS}\\
Wearable, implanted and digestible sensors measure health parameters and vital signs wherever patients or their caregivers need. These sensors help bring healthcare to the homes of patients and make telemedicine and preventive care possible.


\item\textbf{NANOTECHNOLOGY}

In the future, nanotechnology could also enable objects to harvest energy from their environment. New nano-materials and concepts are currently being developed that show potential for producing energy from movement, light, variations in temperature, glucose and other sources with high conversion efficiency



\end{itemize}

\end{document}